\documentclass[11pt]{article}
\usepackage{titlesec}
\usepackage{graphicx}
\usepackage[utf8]{inputenc}
\date{}
\usepackage[export]{adjustbox}
\begin{document}
\title{\line(1,0){250} \\ \huge{\textbf{2019-2020 Bahar Yarıyılı\\ BLM2512 Veri Yapıları ve Algoritmalar 2. Ödevi}} \\ \line(1,0){250}}
\author{\textsc{Mesut Şafak Bilici} \\ \textsc{17011086}}
\maketitle

Bu ödevde bir \textit{Linked List} yapısı kurup, onu \textit{Inserton Sort} ile sıralayıp, \textit{Huffman Tree} yapısına çevirip Level Order bir şekilde bastırmam gerekiyordu. Bu işlemleri yaptığım fonksiyonlar aşağıdaki gösterilmiştir:

\begin{enumerate}
\item void insertLinkedList(NODE**,char);
	\begin{itemize}
	\item İlk olarak input dosyamdaki harfler için frekans bazlı bir linked 			  linked list oluşturmam gerekiyordu. Buradaki önemli kontrol
	      harfleri teker teker yerleştirirken ve tüm linked list'i kontrol       		  ederken, while geldiğinde eğer en sondaysam, ya next NULL olabilir 	 		  ya da son eleman input char ile aynıdır. 
	\end{itemize}
\item void printLinkedList(NODE**);
	\begin{itemize}
	\item Basit bir şekilde, current isimli geçici değişkenimle linked   			      list'in sonuna giderken bastırıyoruz. Program içinde kullanıldığı     		  bir yer yok fakat kontrol amaçlı yazıldı.
	\end{itemize}
\item void insertionSortToLinkedList(NODE**);
	\begin{itemize}
	\item Burada Huffman Tree oluşturmadan önce son işlemimizi yapıyoruz: 				  Linked List'i sıralamak. Verilen ödevlendirme üzerinde kullanılan   		  algoritma Insertion Sort olmuştur. Bilindiği gibi Insertion Sort'ta 
		  kontrol işlemleri için iki farklı indis değişkeni, ve bu indis  		       	  değişkenleri ile iki farklı array bölgesi tutuyoruz. Bu değerleri  			  karşılaştırarak da sıralamamızı yapıyoruz. Fakat elimizde indis 	   		  tutma durumu olmadığı için bu işe yarayan değişkenlerimiz 	  				  \textit{start1} ve \textit{start2} olarak belirlenmiştir. Tıpki  				  basit bir array sıralama işlemi yapıyormuş örneği verilecek olursa, 		  i 1'den başlayacak şekilde start1 = arr[i] ve j=i-1 olacak şekilde
		  start2 = arr[j] olmuştur. En başta belirlenen temp-of-the-temps
		  isimli değişken her bir node'un letter bilgisini swap etmek için
		  tanımlanmıştır. Kullanılan mantık basit bir array işlemi ile 
		  aynı fakat elimizde array indexing yerine linked list vardır.  
	\end{itemize}
\item void readFromFile(NODE**,char*);
	\begin{itemize}
	\item fgetc(FILE*) fonksiyonu ile teker teker karakterler okunmuştur.
		  Fakat dosyanın sonuna geldiğimizde new line okuduğu için her zaman 
		  içerde buffer'ın new line karakterine eşit olup olmama kontrolü 	 		 	  yapıldı.
	\end{itemize}
\item void readFromStdin(NODE**);
	\begin{itemize}
	\item Elle klavyeden girdi yapmak için fonksiyon. EOF sinyali gelene
		  kadar okuma yapmaya devam eder. (EOF = Ctrl+d)
	\end{itemize}
\item void insertButSorted(NODE**,NODE*);
	\begin{itemize}
	\item Bu fonksiyon özel olarak Huffman Ağacını oluştururken toplanmış
		  frekanslara sahip olan node'ların yerini belirleyip oraya insert
		  etmek için yazılmıştır. Eğer direkt head'in frekansından düşükse
		  head'in yerine geçer bir ilerisi eski head olur. Eğer ki öyle 
		  değilse önceden linked list sıralandığı için büyük olana kadar
		  frekans, linked list içerisinde ilerlenir ve sonrasında olması
		  gereken yere insert edilir.
	\end{itemize}
\item void doThatTree(NODE**);
	\begin{itemize}
	\item Bu fonksiyonda artık Huffman Tree'yi tam olarak gerçekliyoruz.
		  Elimizdeki linked list doubly linked list olmadığı için bir önceki
		  node'a ulaşmamız biraz güçleşiyor. Bunu engellemek için iki farklı 
		  current değişkeni tanımladım. While içinde linked listin sonuna
		  giderek ağacımızı gerçekledim. İlk olarak yeni oluşacak node için
		  dinamik bellek alokasyonu yaptım. Sonra bu newnode'un frekans
		  bilgisine linked listin en küçük iki elemanının frekans 
		  bilgisini toplamını atadım. Daha sonra bu newnode'un left ve right
		  bilgilerine bu iki node'u atadım. Bu işlemi linked list'in sonuna 
		  kadar gerçekleştirdim. İstenilen input'un çıktısında boşluk
		  karakteri olduğu için toplam frekansların letter bilgisine * 
		  char'ını girdim. Bu şekilde boşluğun konumunu görebileceksiniz.
	\end{itemize}
\item void printSpecificLevel(NODE*,int);
	\begin{itemize}
	\item Bu fonksiyon verilen level'deki nodeları bastırıyor teker teker.
		  Recursive bir fonksiyon. Eğer girilen NODE* parametresi NULL ise 
		  çıkıyoruz fonksiyondan. Eğer NODE* parametresi tree'nin root'u ise
		  bunu kontrol edip root'un frekansını ve letter'ını (root'un 
		  letter'ı yok fakat karmaşa olmasın diye *'e eşitledik) 
		  bastırıyoruz. Eğer bunlardan hiçbirisi ise recursive bir şekilde 
		  leveli azaltarak node'larımızın left ve right bilgilerini basıyoruz
		  .
	\end{itemize}
\item void printThatTree(NODE*);
	\begin{itemize}
	\item Bu fonskiyonda spesifik level'i bastırdığımız fonksiyonu ve ağacın
		  level sayısını hesapladığımız fonksiyonu birlikte kullanıyoruz.
		  Ağacın uzunluğunu bulup, for döngüsü içinde ve sınır bu uzunluk
		  olacak şekilde teker teker ağacın levellerini bastırıyoruz.
	\end{itemize}
\item int getLevelCount(NODE*);
	\begin{itemize}
	\item Ağacımızın uzunluğunu / level sayısını hesaplıyoruz. Recursive bir 
	fonksiyon. 
	\end{itemize}
\end{enumerate}
\section{Main:}
\hspace*{1cm} Main fonksiyonu içinde teker teker yukarıda belirtilen fonksiyonlar çağrılmıştır istenilenin yapılması için. Öncesinde kullanıcıya elle input mu gireceksin yoksa file'dan mı okuyacaksın sorusu sorulur. File ise file'ın ismini sorar. 
\pagebreak
\\
\hspace*{1cm} \Large{\textbf{İstenilen çıktı aşağıdaki gibidir:}}

\begin{figure}[htbp] % htbp stand for "here", "top", "bottom", "page"
\includegraphics[scale=0.75,center]{aa.png} 
\end{figure}
\hspace*{1cm} 

\end{document}